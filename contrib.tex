This work was supported by a dissertation committee consisting of Professor Randall Stone (advisor), Hein Goemans, and Curtis Signorino of the Department of Political Science and Professor Mark Bils of the Department of Economics.
Graduate study was supported by a Sproull Fellowship from the University of Rochester,
THE Star Lab Fellowship, and a pre-doctoral fellowship from the QuanTM Institute at Emory University.
All work for the dissertation was completed independently by the student.

%  LocalWords:  Hein Goemans Signorino Sproull pre QuanTM
