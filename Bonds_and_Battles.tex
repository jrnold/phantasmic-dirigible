





\section{Introduction}
\label{sec:introduction}



\section{Financial Markets and the Study of War}
\label{sec:barg-theory-war}

Financial markets can play several roles in the study of war in several ways.
First, they provide a measure of the the costs of war \parencites{SchneiderTroeger2006}{GuidolinLaFerrara2010}.
Second, the financial markets are themselves a key actor or mechanism in several theories of war, including the ``Capitalist Peace'' \parencites{Gartzke2007}{DafoeKelsey2014a}, in which markets signal the costs of war, or \textcite{Slantchev2012a} in which debt financing creates a commitment problem.
Third, in some cases, the prices of financial assets act similar to a prediction market of the expected onset or outcome of war.
The implied predictions or expectations of the war outcome derived from the prices can be used to assess theories of war initiation and termination.%
\footnote{See Chapter~\ref{cha:financ-mark-onset} for an analysis of how financial markets assessed the probability of the onset of the American Civil War.}
In this work, I focus on the third case, and use the yields of U.S. government bonds to measure expectations of the outcome of the American Civil War.%
\footnote{
  I use the term ``outcome'' to mean the state of the world at the end of the war, including the duration, cost, and victor of the war.
  I use the term ``result'' to refer to the specific outcome of who ``won'' or ``lost'' the war, which can be context specific.
}

\subsection{How the Prices of Financial Assets Relate to War Outcomes}
\label{sec:how-prices-financial}

A financial asset is a claim on a stream of future, possibly uncertain, cash flows.%
\footnote{
  This discussion largely follows the discussion in \textcite[673]{GuidolinLaFerrara2010}; refer to them for more detail.
  See also \textcite{HaberMitchenerOosterlinckEtAl2015}.
  See any introductory finance textbook or course notes for more information.
  See \textcite{Chan-Lau2006} for an overview of market-based measures of sovereign risk.
}
For example, a coupon bond pays a set number of coupons at specified times and its face value on maturity.
A stock pays dividends at potentially uncertain times.
Both because these cash flows are in the future and because their payment may be uncertain, these cash flows are discounted.
The current price of a financial asset ($P_{t}$) is the value of those discounted cash flows,
\begin{equation}
  \label{bonds:eq:3}
  P_{t} = \sum_{j = 1}^{H} \frac{\E_{t + j}\left(C_{t + j} | \mathcal{I}_{t}\right)}{1 + \E_{t}\left(\delta_{t+j} | \mathcal{I}_{t}\right)}
\end{equation}
where $H$ is the number of cash flows, $C_{t + j}$ is the cash flow at time $t + j$, and $\delta_{t + j}$ is the discount rate for time $t + j$, and $\mathcal{I}_{t}$ is the information available to the market at the current time $t$.
The discount rate often consists of a risk free interest rate and a premium specific to that asset, both of which may be time-varying.
The riskiness of the asset can also be incorporated in $\E(C_{t}| \mathcal{I}_{t})$.
For example, if the if there is a non-zero probability of default $\E(C_{t})$ would be a function of the probability of default and the amounts received in default or non-default states.
The key feature of Equation (\ref{bonds:eq:3}) is that all of the price inputs are expectations conditioned on current information.
Thus, events can change the price through changing the market's information set used to formulate expectations about future events. 

Since the prices of financial assets are a function of expectations about future cash flows and risk premia, assets in which the those cash flows or risk premia are largely contingent on some outcome of a war are themselves a proxy for expectations about that war outcome.
For this purpose, the ideal asset would be a binary option, which pays out a non-zero amount if the war outcome of interest occurs and zero otherwise.
Then, with only some assumptions about the risk-preferences of the market, the probability of the war outcome is easily calculated from the price of the asset.
Prediction markets often use binary options, and there have been several examples of prediction markets for political events which have used binary options.
The Iowa Electronic Market is a prediction market for U.S. presidential and Congressional elections.%
\footnote{
  See \textcite{WolfersZitzewitz2004} for an overview of prediction markets.
  \textcite{RhodeStrumpf2004a} discusses historical betting markets on U.S. presidential elections in the late nineteenth-early twentieth century, which unfortunately for me were not formalized until 1884.
  See \url{http://tippie.uiowa.edu/iem/} for the Iowa Electronic Market.
}
Intrade, until its closure in 2013, had prediction markets for multiple political and foreign policy events.
These prediction markets included U.S. and other election, whether Saddam Hussein would be removed as the president of Iraq, when Osama Bin Laden would be either killed or captured, and when a there would be a U.S. or Israeli airstrike against Iran.%
\footnote{
  The Saddam contract issued by Tradesports is used in \textcite{LeighWolfersEtAl2003}.
  See \url{http://intrade-archive.appspot.com/event.jsp?event=4272}, \url{http://intrade-archive.appspot.com/event.jsp?event=37985} Intrade contracts on Osama bin Laden and an airstrike against Iran, respectively.
}
Unfortunately for the researcher of conflict, these prediction markets in political events are not widespread at present, and not available historically for wars.
However, financial assets in which the cash flows or risk premia are contingent on a war outcome are effectively prediction markets for that war outcome.
In particular, the sovereign bonds issued by the belligerents in a war are likely to be highly sensitive to war outcomes.
These bonds will in general respond to two features of the war outcome.
First, the riskiness of a belligerent's sovereign bond is a function of the expected cost of the war, since a more costly war implies a higher cost of war.
A more costly war almost directly implies a higher debt or higher taxes that cannot be used to service existing debt leading to either a higher tax  a higher probability of default for the bond (lower price).
Wars are also costly in more ways than just the direct government expenditures.
The destruction of human capital, military and civilian casualties, and physical capital effects on the expected economic growth of that country, and its future resources with which to pay off the debt.%
\footnote{See \textcite{GoldinLewis1975} for a calculation of the costs of the American Civil War.}
Note that these expectations of the cost of war will incorporate both expectations about both the intensity of the war and its duration.
Second, the riskiness of a belligerent's sovereign bond is a function of whether the belligerent is expected to win or lose the war.
Victory in war can gain the belligerent more territory and greater resources to pay off debt; a loss can mean the converse.
In some cases, a defeat can lead to a loss of sovereignty and a default on their debt.
This is relatively rare in inter-state wars, but more common in civil wars in which the rebel side issued debt \parencite{HaberMitchenerOosterlinckEtAl2015}.
This was the case with the Confederate States.% TODO: cite something on loss of so
\footnote{Section 4 of the Fourteenth Amendment, passed after the American Civil War, explicitly repudiates Confederate debt.}
More generally, losing states may tend to default on their debt with a higher probability \parencite{Slantchev2012a}.
The price of a bond is some weighted function of all of these expectations about what the outcome of the war will be and how those outcomes will effect the ability of the belligerent to repay their debt.
For some bonds, it may easy to back out the probability of a specific war outcome.
\textcite{HaberMitchenerOosterlinckEtAl2015} focus on cases in which rebels issued bonds in civil wars (American Civil War, Chinese Civil War, and Spanish Civil War) and estimate the probability that the rebel side will win.
% Another method is by using the prices of several assets from different sides, it may also be possible estimate expectations of victory and defeat; see \textcite{Hall2004} and \textcite{McCandless1996}.
But it may not be possible to disentangle expectations about the cost and outcomes of the war for the belligerents.
However, there is some advantage to the way in which financial assets weight these war outcomes.
By putting all war outcomes on the same scale, namely, their fiscal effect on the belligerent(s), the prices of sovereign bonds are able to provide a single, if incomplete, measure of the war outcome for each belligerent that incorporates both the costs and benefits of the war to that belligerent.
Additionally, because interest rates are expectations and are generally available at a high frequency, they provide a real-time measure of expectations about the war.
Sovereign bonds are not the only financial assets which could be used as a proxy for war outcomes, stock market indices or specific sectors thereof \textcites{ChenSiems2004}{SchneiderTroeger2006}{WolfersZitzewitz2009}, exchange rates \textcites{Hall2004}, and commodities \textcites{WolfersZitzewitz2009}, may respond to conflict in some cases and may be able to be used as proxies for expectations about war outcomes.
However, while there are commonalities, the specifics of how a financial asset relates to war outcomes is not universal, and in each case the researcher should carefully consider how the financial asset relates to the war outcome of interest.
When using a financial asset price fas a proxy for an expected war outcome, it is also necessary that its price be primarily influenced by that war outcome and not other factors.
For example, when studying the Iraq War, it would be implausible to use the interest rates of U.S. Treasury bonds or the S\&P 500 index since the effect of the war one way or another is likely a small influence on them.%
\footnote{
  It is still possible to ask questions such as how much influence did events related to the Iraq War have on the stock market, as in \text{WolfersZitzewitz2009}.
  It would just be difficult to ask the inverse question which may be of more interest to political science research; given that the stock market is proxying expectations of the Iraq War, what events had the largest influence on it.
 }
It is plausible to use bonds issued by Iraq, if they had issued in 2003, because the outcome or initiation of a war with the U.S. would plausibly be the most important factor influencing them.
This is not out of concern of confounding per se, because factors that are not correlated with the war outcome can also be a problem.
The concern is that the signal (war outcome) to noise (other factors that cannot be controlled for) ratio of the price needs to be large for it to be plausible to use the price as a proxy for a war outcome.
Using sovereign bonds, or most financial assets, as proxies for expectations of war outcomes will only be able to proxy for war outcomes so long as those outcomes has some influence on either cash flows or risk premia.
This means that some policy objectives of a war, \eg{}national pride or the abolition of slavery in the American Civil War, may be important to the belligerents and of interest to the researcher, but if they do not affect future cash flows or risk premia, then the prices of financial assets have little to say about them.

In this work, instead of prices, I use the yields to maturity of coupon bonds.
The yield to maturity of a coupon bond can be found by taking a known price and cash flow schedule in Equation (\ref{bonds:eq:3}) as solving for a constant discount rate.
For example, assuming continuous compounding for simplicity, the yield to maturity, $y$, of a bond is the solution to
\begin{equation}
  \label{bonds:eq:5}
  P_{t} = \sum_{j = 1}^{H} C_{t + j} e^{-y j} \text{.}
\end{equation}
Equation (\ref{bonds:eq:5}) shows that the interest rate moves inversely to the price: a higher yield corresponds to a lower price, and vice-versa.
In other words, the riskier a bond is, the lower the price and the higher the yield.
%I use the yield to maturity of the bonds rather than their prices,  because prices deterministically drift towards par with time and because they are more interpretable.
% interpretation of risk in terms of default




\subsection{How Financial Markets Can Improve the Study of War}
\label{sec:how-prices-financial-1}

Since \textcite{Fearon1995}, the the dominant theory of war has been the bargaining theory of war.%
\footnote{See \textcites{Reiter2003}{Powell2006}{Reiter2009} for discussions.}
While there is a large theoretical literature that has formalized much of the theory, \parencites{FilsonWerner2002}{Slantchev2003}{SmithStam2004}{Powell2004}{LeventogluSlantchev2007}{LangloisLanglois2009}{WolfordReiterCarrubba2011}, direct empirical evidence of the theory are still limited \parencite{Reiter2009}.
The prices of financial assets can improve the discipline's understanding of war in two ways: expanding the number of cases which can be used for analysis by allowing for intra-case quantitative analysis, and providing a measure of surprising events.

The bargaining theory of war is essentially a formalization of Clausewitz's famous dictum: ``War is merely the continuation of policy by other means.'' \parencite[87]{Clausewitz1989}
The bargaining theory of war sees war as part of a bargaining process.
The central puzzle of war is, given that war is costly, why did the sides not reach an agreement without incurring the costs of war.
In particular, at the end of a war, some agreement is reached, even if that involves the complete capitulation of one of the sides.
So why was that agreement not reached prior to the war and without incurring the costs of the war.
\parencite{Fearon1995} and \parencite{Powell2006} propose two frictions that would prevent an \textit{ex ante agreement}: private information and commitment problems.%
\footnote{
  War can also occur due to the sides the being risk loving or due to domestic politics.
  \textcite{Fearon1995} dismisses risk lovingness as implausible.
  Many of the domestic politics explanations are explanations that generate a private information or commitment problem that allows for conflict as a solution to the bargaining problem.
}
But this leads to a second puzzle, given that these frictions can lead to conflict instead of a peaceful bargain, how does the actual fighting in the conflict resolve that friction so that the sides can come to an agreement and return to peace.
\textcite[757]{LeventogluSlantchev2007} describe theories which answer this question as \textit{complete} and \textit{coherent} theories of war: ``[A theory of war] must be complete, which means it must account for war’s outbreak and termination; and it must be coherent, which means that its account of termination must explain how fighting has resolved the cause.''
So while some evidence for the bargaining theory of war can come from studying war onset, testing the ``core of the theory'' requires intrawar data and as such has demanding data requirements, as such has been subject to few empirical studies \parencites{Reiter2003}{Ramsay2008}{Reiter2009}{Weisiger2015}.
Empirical evaluations the bargaining model of war requires estimates of capabilities, estimates of resolve, exchange of offers, and combat data \parencite{Reiter2003}.
All of these data are generally difficult to acquire for one case, let alone for multiple cases.
Financial asset price data is useful in that it can allow researchers to make the most out of the limited data that is available.

First, financial asset price data allows for within-case quantitative analysis of war termination by using the price of financial assets.
Instead of using inter-case variation in war outcomes in the analysis, financial asset prices provide measures of expected war outcomes with intra-case variation.
Being able to use intra-case variation is useful to researchers, because, apart from controlling for case specific attributes, the data demands of the bargaining theory of war means means that there are few cases with the requisite data.
While data on the sides' estimates of resolve and capabilities and (often secret) bargaining offers are unsurprisingly scarce, data on battles is also of surprisingly poor quality and quantity.
While there exists a Correlates of War for wars, there does not exist a similar dataset for battles, and the available data on battles is limited in both quantity and quality.
The most commonly used and comprehensive battle dataset for inter-state wars is the CDB90 or HERO dataset \parencites{HistoricalResearchEtAl1984}{cdb90}, but that data is both dated and of questionable quality \parencites[32]{Reiter2003}{BiddleLong2004}.
This lack of available data was one motivation for the use of the American Civil War in this work, since it is a war for which data on battles is available of relatively high quality and comprehensiveness.
\footnote{
More recently work by \textcite{Weisiger2015} and \textcite{CochranLong2014} have offered multi-war data.
See \textcites{cdb90}{Helmbold1995} for a discussion of the difficulties in compiling battle data.
}
Even if battle data were more plentiful, the ability to compare features of battles across wars may be fundamentally limited due to the heterogeneity of combat across time and space \parencite{Reiter2009}.
As such, \textcites{Reiter2003}{Reiter2009} suggest that the best methodology for the empirical analysis of the bargaining theory of war is qualitative analysis on limited cases.
But in within-case analysis, some important dependent variables it cannot be used to test any implications on the war outcome, \eg{}duration, since only one war outcome is observed for each war.
But, as discussed in Section~\ref{sec:how-prices-financial}, the prices of some financial assets may act as measures of the expected war termination.
This allows researchers to conduct within-case quantitative analysis using cases with the requisite battle data, and use the prices of carefully-chosen financial assets as an outcome variable that proxies for war outcomes, such as the duration, cost or victor.
This methodology exploits within-case variation in the expected war outcome rather than between-case variation in the observed war outcomes.

Second, financial asset price changes naturally provide a measure of surprising events and new information.
Surprising events are particularly important in informational theories of war, as only new information will affect beliefs.
However,  it is difficult \textit{ex ante} to know what was surprising to contemporaries, see \textcite{Shirkey2009a} for an attempt to code surprising events.
Prices provide a natural measure of new information, since as discussed in Section~\ref{sec:how-prices-financial}, prices are  expectations of future events given the current information available to the market, and changes in prices correspond to changes in the news available to the market, 

The prices of financial assets have several other characteristics that make them a useful addition to the conflict researcher's toolbox for analyzing war.
\footnote{See \textcite{WillardGuinnaneEtAl1996}, \textcite{north2000introd}, and \textcite{FreyKucher2000} for other  discussions on the use of price data to analyze historical events.} %
First, they incorporate only \textit{ex ante} information.
This distinguishes price data from sources written after the fact, including participant memoirs fact or secondary sources \parencites[1001]{WillardGuinnaneEtAl1996}[][188]{FreyKucher2000a}.
Those sources can suffer from hindsight bias because the author is aware of future events.
Second, investors have strong incentives to be truthful in their assessments because their payoffs are contingent on the accuracy of their assessments.
This distinguishes prices from other sources such as the surveys or diaries of decision makers, which like prices have only the information available at the time, but in which the author may have strategic incentives to deceive \parencite[57]{Reiter2009}.%
\footnote{
  Although the investors in the American Civil War had their own preferences on the war, their investments seemed motivated by their preference for profits.
  \textcite[5,7]{Cornwallis1879} writes,
  \begin{quote} 
    Sectional feeling often entered largely into the bull and bear  contests in the Gold Room, and Union men and rebel sympathizers fought their battles sometimes, as much to gratify this as to make money.

    The heaviest speculative orders were sent from Washington and Baltimore, and next to these, from Louisville, Kentucky, owing to these cities being in close communication to the seat of war and the rebel lines; and the operators there, almost to a man, were ``bulls'' in feeling, and strong Secessionists.
    But though seldom or never found selling ``short'' they were quick to sell out their ``long'' gold --- that is the gold they were carrying --- whenever the Confederate arms met with a reverse, and as quick to buy it back again when the market seemed to ``touch bottom''.
  \end{quote}
  \textcite[210]{Mitchell1903} writes,
  \begin{quote}
    Viewed in a broad way, it is therefore a serious mistake to look on the gold market as a place where a few gamblers were tossing the premium about to suit their selfish schemes; a much saner view is that it was the place where the community's estimate of the government's credit was visibly recorded.
    Here, as in other markets, those operators succeeded who forecast the future correctly, and men who tried to advance the price of gold when public confidence was increasing, or to depress it when confidence was on the wane, learned to their cost that they were not masters of the situation.
  \end{quote}
} %
Finally, price data is often available at high frequencies, often daily.
This is in contrast with documentary evidence, which is often either produced at lower frequency or missing for large periods of time \parencite[][57]{Reiter2009}.
The high frequency of price data provides more data points to the researcher, and expands the set of questions that can be answered.
%% Fourth, the prices of financial assets are a quantitative summary of investors' expectations, and less susceptible to the  subjective interpretation of researchers.
%% This distinguishes price data from documentary evidence which must be interpreted by researchers \parencite[][58]{Reiter2009}.
%% The quantitative nature of the data makes it easier to analyze not only what events mattered in conflict, but the magnitude of those effects \parencite{north2000introd}.

This is not to claim that prices of financial assets dominate other data sources.
They simply provide another tool for conflict researchers.
Financial asset prices have several shortcomings when studying conflict, and are not appropriate in all cases.
First, these prices do not directly measure the beliefs of the belligerents.
Prices of financial data generally incorporate public information, and do not incorporate the private information available to decision makers.
However, the causal mechanism in informational theories of war are is the beliefs of the decision makers.
Yet, these means by which those beliefs converge is that conflict reveals private information, making that private information public.
So it may be possible that many private information models of war could restated so that they have implications on changes in the expected outcome of the war conditional on public information, for which these prices are a measure.
Second, an appropriate financial asset may not be available in many conflicts.
In many areas in which researchers are most interested in understanding conflict, \eg{} civil wars in developing countries, a financial asset  that is closely tied to the war outcome may not exist.
Even now, not all developing countries issue sovereign bonds.
Additionally, those countries prone to war may not have liquid markets, so the prices of available assets may not be accurate assessments of expectations.
Even states with liquid and well-developed markets in peace-time, states often intervene in their own markets during war which distorts the price signals of those assets \parencite[12]{HaberMitchenerOosterlinckEtAl2015}.
Finally, in any case in which prices are used, they cannot be used without domain specific knowledge.
The researcher will need to apply their substantive knowledge of the assets, market, and conflict in order to find a financial asset that is related to expected war outcomes, and to understand how that asset relates to the war outcomes which are of interest to the researcher.



\subsection{The Means and Mechanisms of War}
\label{sec:means-mechanisms-war}

While I motivate the use of financial asset prices to study war from the perspective of the bargaining theory of war, I do not attempt to directly test the private information and commitment theories of war in this work.%
\footnote{See \textcite{Ramsay2008}, \textcite{Weisiger2015}, and \textcite{Reiter2009} for works which attempt to do so.}
Instead, I focus on the effects of military combat --- victories and defeats in major battles --- on the expected cost of the war.
I do not attempt to distinguish whether these victories and defeats influence the expected war termination through the revelation of private information or through resolving commitment problems.
I do this because distinguishing between private information and commitment theories of war using only battle data is harder than previously appreciated.
Observing only battle results, such as victories or casualties, is insufficient to identify the casual mechanism of the war without relying on a well-specified model of the war process.%
These models do not exist at the moment.
Models of that kind would require a deeper study of military processes, which by estimating the effects of battles on war termination, and identifying which battles seemed to have the largest effect, this work may in small part contribute.
This section explains the reasoning behind this claim.

In the bargaining theory of war, war combines a political bargaining process with military combat process. %
\footnote{%
  ``Combat is a violent clash between at least two politically
  distinct groups organized to wield force. War consists of sustained
  and substantial episodes of combat.'' \parencite{Reiter2003} %
} %
These processes are distinct yet interconnected.%
For example, in costly process formal models of war, war is modeled as alternating rounds of ``bargaining'' and ``fighting''. %
The bargaining and combat are distinct, but the outcome of the game depends on both.
As mentioned in the previous section, a complete and coherent theory of war needs to explain how fighting, \ie{}military combat, resolves the friction---private information or commitment problems---that prevented an agreement without war.
So means of war is military combat, the immediate goals of which are the destruction of military forces, the destruction of civilian assets, or the control of territory.
\parencite[30]{Reiter2003}.%
\footnote{
While formal models of war can be categorized by whether they are models of the private information mechanism, commitment problem, or both, they can also be categorized by how they model military combat within the war.
In some models battles are important primarily because they influence the probability of war victory, \parencites{Powell2004}{Wagner2000}{LeventogluSlantchev2007}{Slantchev2003}{SmithStam2004}.
In other models, battles are important because they impose costs on the sides \parencites{FilsonWerner2002}{Powell2004}{LeventogluSlantchev2007}. %
}
These means of war must lead to war termination by resolving either the private information or commitment problems in some way.
But there may exist immediate means of combat and multiple channels by which each of those means may influence war termination.
Figure~\ref{bonds:fig:combat-causal-diagram} illustrates the relationship between means, mechanisms ans war outcomes.
In one view, resolution of private information or the commitment problem are causes of war termination.
In another view, military combat is a cause of war termination and resolution of private information and commitment problems are mediators.
In general, it is dificult to identify the causal effects of mediators \parencite{Keele2015a}.

\begin{figure}[!htpb]
  \centering
  \input{war_psuedo_casual.tikz}
  \caption{Combat and War Termination: A Causal Model}
  \label{bonds:fig:combat-causal-diagram}
\end{figure}

With regard to war, identifying the channels through which combat influences the war termination is empirically quite difficult.
For almost any observable data on military combat, it both alters the actual state of the war (potentially resolving a commitment problem) and the beliefs of the participants about the combat process in which they are involved.
There are a couple ways in which this effects might be identified. %
First, if the beliefs of the participants were observed, then the effects of war events on the beliefs of the participants could be measured. %
A problem with this approach is that it requires data on the beliefs of the participants, which is difficult to acquire.
Even if first-hand accounts are available, it is not clear whether these are accurate indicators of those beliefs. %
Another problem, is that it is insufficient to simply observe changes in the beliefs of one participant over time, as in \textcite{Reiter2009}. %
The causal mechanism in informational theories of war is the convergence of beliefs of the belligerents to a common distribution, to it requires knowing the beliefs of both sides.
Second, it could be identified if commitment problem that started the war were known and changes it were observable. 
Third, identification would require a detailed model of how military combat influences war outcomes.
Then it may be possible to back out how belligerents would update their beliefs and how combat could result in military victory or resolve commitment problems.
However, this would require more work along the lines of \textcite{Biddle2004}, which focus on generating models of military combat correct, even if these models are not directly tied to political variables.
Thus to understand war termination it is necessary to understand the details of the military combat process that underlies the bargaining model. %
The difficulty is that these processes may be extremely heterogeneous, across time and space \parencite{Reiter2003}.
But these are not reasons to ignore the study of military combat, but a call to describe and explain those patterns.
Better understanding of the the military combat processes in war may have different implications for these models, and thus make it easier to distinguish when or if they are present.%
\footnote{%
  For example, \textcite{Walter2009} conjectures that insurgency is slower to reveal information than other forms of war. %
  Rigorously evaluating theories in this manner would require better theories of the actual military rationale between insurgency versus conventional war, the characteristics of each, and when belligerents choose to use each.
} %



\section{Why The American Civil War?}
\label{sec:why-american-civil}

The American Civil War is a war with plentiful, and high quality battle-level data, and in which there are financial assets which proxy expected war outcomes well.
In terms of its battle data, the American Civil War is one of the best documented wars, so quality battle-level is available.%
\footnote{%
  For example, \textit{The Official Records of the Union and  Confederate Armies} \parencites{US1901}, published between 1880 and 1901, consists of 128 books in 70 volumes totaling 139 thousand pages.

} %
In Section~\ref{sec:why-prices-study}, I discuss why financial assets issued by the belligerents of the American Civil War may be good proxies for its war outcome.

Like any case, its generalizability is limited by its similarity to other observations.
But despite its age, the American Civil War has several similarities to contemporary civil wars.
First, the American Civil War is, if not the first modern war, a wars in the transition to modern warfare.
This war introduced several of the technologies which came to prominence in World War I and later wars: rifled artillery, telegraph, machine guns, barbed wire, ironclad warships, and submarines \parencites[89][]{Fuller1956a}[760]{Weiss1966}.
Second, the American Civil War was primarily a conventional war.
This places it in the plurality of post-Cold War civil wars, which are more often conventional wars than insurgencies \parencite[423]{kalyvas2010inter}. %
Finally, by the Civil War was U.S. was industrializing and the purchasing power parity adjusted GDP per capita of the U.S. and Confederate States in 1860 would classify them them as lower-middle income countries today.%
\footnote{%
  The GDP per capita of the US in 1860 was \$2,241 in 1990 GK international dollars. %
  In 2010, Cambodia had a GDP per capita of \$2,450, Pakistan \$2,494, and Ghana \$1,922 \parencite{BoltZanden2013}. %
  The southern states were poorer than the northern states, with a per capita consumption of about 70 percent of the overall U.S. level \parencite[324]{GoldinLewis1975}.
  The GDP per capita of the pre-war Confederacy is similar to that of Angola, Iraq and Senegal in 2010.%
  The estimated GDP per capita of the southern states in 1860 at 70\% that of the northern states was \$1,568. %
  In 2010, the following countries had approximately the same real GDP per capita: Angola \$1,600, Iraq \$1,610, and Senegal \$1,507 \parencite{BoltZanden2013}. %
}

The American Civil War is also an interesting case for study in its own right.
Its importance to American history is obvious, and need not be stated.%
\footnote{See \textcite{McPherson2003} among too many others to list.}
But, given the voluminous literature on the subject, it is surprisingly understudied in at least two regards. %
First, the international relations literature has largely ignored this conflict.
While the inter-state war literature considers it a civil war, and the civil war literature focuses on the post-1945 era \parencites[140-141]{Reiter2009}[2]{Poast2012}. %
Exceptions to this are \textcite{Reiter2009} and \textcite{Poast2012}.
Second, almost none of the existing work on the American Civil War conflict, including the few examples in international relations, use quantitative methodologies. %
\footnote{\textcite{Weiss1966} for an example from operations research.}
In applying international relations theory and quantitative methods to the study of this war, this work contributes a droplet to the ocean that is the literature on the American Civil War.



\section{U.S. Government Bond Yield Data}
\label{sec:why-prices-study}



\subsection{The Fives of 1874}
\label{sec:5s-1874}

I will use the yields of long-run U.S. government bonds as a measure of the expected outcome of the American Civil War.
While the yields of bonds of the U.S. government are almost certain to have been primarily driven by expectations of the war, they could plausibly be influenced by both expectations of U.S. victory and expectations of the cost of the war.
However, it is likely that the cost and duration of the war is the more important of those factors.
However, using the yield of U.S. government bonds is difficult since while the U.S. issued a large amount of debt during the American Civil War, the U.S. issued relatively little debt in the ante-bellum period.%
\footnote{
  After paying off the debt issued during the War of 1812, the U.S. government had no outstanding debt between 1835 and 1842 \parencite[297]{HomerSylla2005}.
  The debt as a percentage of GDP rose from 1.9 percent in 1860, before the war, to 31.4 percent in 1866, its peak just after the war \parencites{CBO2012}{CBO2012a}.
}
In this work, I use the the yields of the U.S. Fives of 1874, because it is the only government bond that was both issued prior to the war and quoted throughout the war, meaning that it is the only complete time series from a single asset that spans the war.

The Fives of 1874 were a coupon bond issued under the authorization of the Act of June 14, 1858 to pay for a budget shortfall due to the Panic of 1857, a financial panic and recession.%
\footnote{
  See \textcites[p. 76]{Bayley1882}[78--79]{DeKnight1900}[42-43]{Treasury1863}[300-301, 305]{HomerSylla2005}
}
As their name suggests, the Fives of 1874 were a coupon bond that paid 5 percent semi-annually on January and July, and were redeemable in 1874, meaning it had a maturity of 13--9 years during the period considered here.
Like most U.S. government bonds, the Fives of 1874 paid interest and principal in specie (gold dollars), even after the U.S. Treasury ceased the redemption Greenbacks (U.S. non-interest bearing notes used as currency issued during the war) in gold.
The U.S. currency floated against the gold dollar until the resumption of convertibility on January 1st, 1879 \parencites{Dewey1918}{WillardGuinnaneEtAl1996}.
For this analysis the prices of the bonds will be converted to gold dollars at the current exchange, and the yields of the bonds calculated with the cash flows converted to gold dollars accounting for the current exchange rate and the expected depreciation of the gold dollar.%
\footnote{
  Since the price of bonds was quoted in currency but it paid interest and principal in currency, calculating the yields is difficult since they also incorporated beliefs about the future prices of gold dollars \parencites[Appendix A]{Macaulay1938}{Roll1972}{Calomiris1988}[302-303]{HomerSylla2005}.
  Most bonds, including the Fives of 1874, were priced in U.S. currency but paid specie and often principal in gold dollars.
  This work accounts for this using a simple method to calculate the future price path of the dollar; it assumes that currency with appreciate at the risk free interest rate of 5 percent until it reaches parity with a gold dollar.
  This follows from an assumption that the risk free rate is constant over the period, and the gold dollar is priced as a risk-free zero-coupon bond with an unknown redemption date \cite{McCandless1996}.
  \textcite{Roll1972} and \textcite{Calomiris1988} estimate the expected depreciation of gold dollars in Greenbacks using differentials in bonds, however those methods cannot be applied to the period considered here.
}
As will be shown, the price and yields calculated in terms of gold dollars fluctuated dramatically throughout the war.
Yet, the price and yield in terms of currency (after January 1, 1862 when the Greenback floated against the gold dollar, were mostly stable.%
\footnote{See \textcite{Roll1972} for an analysis of this phenomena.}
This suggests that investors were less worried about a complete default on payments of the bonds than that the government would redeem interest or principal payments in depreciated currency rather than in specie.

Figures~\ref{bonds:fig:fives1874_price} and~\ref{bonds:fig:fives1874_yield} plot the price in gold dollars and the yield to maturity of the Fives of 1874.
Prior to mid-1860, the yield of the Fives was low with little variation, fluctuating between 4.5 and 5 percent.%
\footnote{
  This is actually below was generally considered risk-free rate in the U.S. for that period, 5 percent, which \textcite[29]{Elder1863} calls the ``natural rate of interest on [government] Loans''.
  During this period British 3 percent consols had a yield of between of 3.10 and 3.28 \parencite[193]{HomerSylla2005}.
}
After Lincoln's election in November 1860,  the interest rate rose to 6 percent.
After the initiation of the war with the Battle of Sumter (April 12--14, 1861), the yield rose to 8.2 percent.
During the war, between April 1861 and April 1865, the yield fluctuated between 5.2 and 14.7 percent, reaching its peak on July 30, 1864.
After the end of conflict in April 1865, the interest rate returned to stability at 6.5 percent, a rate higher than its pre-war level.

Data on the prices of the Fives of 1874 come from tables in the two leading monthly financial magazines of the era, the \textit{Bankers' Magazine and Statistical Register} and the \textit{The Merchants' Magazine and Commercial Chronicle} \parencites[186]{Mitchell1903}.
In both sources, the quoted prices were for the New York market, and prices were in currency.
I converted Greenbacks to gold dollars using the current exchange rate in New York, using the data of \textcite{Mitchell1908}.%
\footnote{
  Available in digital form at \url{http://eh.net/wp-content/uploads/2013/11/greenback.txt} from \textcite{WillardGuinnaneEtAl1996}.
  I made some corrections to it where there were incorrect transcriptions from the original source.
}
Both sources quote the bonds at irregular, but approximately weekly frequencies: \textit{Bankers' Magazine} (3--15 days, median of 10 days), \textit{Merchants' Magazine} (4--23 days, wit a median of 7 days).
For the analysis, the quotes of these two series are combined into a single series.

I collected and provide the raw financial data used in this work, as well as additional financial and economic from the American Civil War period in the American Civil War Era Financial Data Collection (ACWFD) \parencite{Arnold2015a}.%
\footnote{Also available at The dataset is at \url{http://github.com/jrnold/civil_war_era_findata}.}
This data collection contains bond prices and exchange rates published in the \textit{Bankers' Magazine} and \textit{Merchants' Magazine}, bid-level auction data for U.S. government bond auctions, and miscellaneous financial and economic data for this period from a variety of primary and secondary sources.

The Fives of 1874 were one of several bonds issued by the U.S. government during this period, but it has several advantages over other bond yield series.
The Fives of 1874 was the only coupon bond regularly quoted in the sources considered here for the entire war, April 1861 to May 1865.
Only one other bond was widely traded at the start of the war, the Sixes of 1848--1868.
However, \textit{Bankers' Magazine} stops quoting the Sixes of 1868 in September 1861.
Another bond, the  Sixes of 1861--1881 were issued under acts in February and July 1861, but they are not regularly quoted in the \textit{Bankers' Magazine} until September 1861.
The U.S. issued several other bonds of various durations and coupon rates throughout the war, but they provide even shorter time series.
Additionally, in most cases these loans had relatively exotic features, such as an option to exchange for another bond series or being callable by the government, which make it difficult to calculate the yields for them.
\footnote{
  The 5-20's and 10-40's were bonds callable by the government after 5 (10) years and redeemable in 20 (40) years.
  The 7-30's were bonds that had coupon rates of 7.30 percent (2 cents per day), were redeemable in 3 years, but included the option to be exchanged for either the Sixes of 1881 or 5-20's.
  See \textcites{Bayley1882}{DeKnight1900}[297--309]{HomerSylla2005} for overviews of U.S. debt issues during this period.
}
Using the Fives of 1874 relative to other bonds should not make a substantive difference, since as Figure~\ref{bonds:fig:fives1874_union} shows, the yield to maturity of the Fives of 1874 is similar to the yields of all the other bonds issued by the U.S. government.
\textcites{WillardGuinnaneEtAl1996}{McCandless1996}{SmithSmith1997} use the exchange rate between Greenbacks and gold dollars in their analyses.
However, since the U.S. did not issue currency until after the start of the war, and did not suspend the redemption of that currency for gold until January 1, 1862, any analysis using only Greenback prices cannot cover the first eight months of the war, April--December 1861.
This is the period which should have the highest informational content in an informational theory of war, and includes battles such as the First Battle of Bull Run, and the Battle of Wilson's Creek.%
\footnote{\textcite{Poast2012} argues that the First Battle of Bull Run was pivotal in avoiding British recognition of the Confederate States.}
However, as with the Fives of 1874 versus other bonds,  Figure~\ref{bonds:fig:fives1874_greenbacks} shows that the prices of the Fives of 1874 in gold dollars is similar in both level and trend as the the price of Greenbacks in gold dollars, so the choice is unlikely to change the results.

The American Civil War is an ideal case in which to use prices to infer expectations of war termination for three reasons.
First, The war was of such severity that the nearly the entire budget of the U.S. government was war-related military spending, and thus fiscal expectations corresponded to military spending expectations. %
For the U.S., the War and Navy Departments accounted for 76 percent (in 1862) to 61 percent (in 1865) of the expenditures.
The fraction of the budget spent directly on the military by the Union fell over the course the war as the fraction spent on paying the principal and interest on its debt, increased from 20 (1862) to 36 (1865) percent of the budget.
But, the overwhelming the majority of that debt had been issued to fund the war \parencites{Treasury1861a}{Treasury1861b}{Treasury1862}{Treasury1863}{Treasury1864}{Treasury1865}.
As \textcite[][668]{McCandless1996} states, ``[d]uring the war, the single most important indicator of future government expenditures is the process of the war itself.''
% INSERT FIGURE OF THE BUDGET

Second, there is extensive economic history literature establishing that war events are the single best determinant of bond and currency prices during the American Civil War:
\parencites{Mitchell1903}{Mitchell1908}{Calomiris1988}{WillardGuinnaneEtAl1996}{McCandless1996}{SmithSmith1997}, graybacks \parencites{Schwab1901}{Weidenmier2002}, southern prices
\parencite{BurdekinLangdana1993}, Confederate bonds \parencites{DavisPecquet1990}{BrownBurdekin2000}{OosterlinckWeidenmier2007}, and Union bonds \parencite{Roll1972}.%
\footnote{One notable exception is \textcite{BurdekinWeidenmier2001}, which shows that the prices of graybacks in Richmond and Houston diverged due to differential application of monetary reform in late 1864.}
%These are largely interested in showing that the markets efficiently responded to new information in the war; this work is inverting that, and starting with the assumption that the markets are efficiently processing new information to understand imporant

Third, statements by and the actions of contemporary investors indicated the importance that war events had on the prices. %
Members of the government, military, and reporters all engaged in investment speculation based on their knowledge of war events \parencites[5-7]{Cornwallis1879}{Mitchell1903}[][1004]{WillardGuinnaneEtAl1996}.
Investment firms even had their own correspondents stationed in cities close to the war fronts to relay the latest news \parencites[5-7]{Cornwallis1879}. %
Finally, magazines and newspapers often cited war events as the cause of recent price movements. %
For example, the issue of \textit{Bankers' Magazine} after Gettysburg and Vicksburg (Jul 23, 1863), cites these battles as the cause of the recent changes in market prices,
=\begin{quote}
  The month of July has been a very active one with numerous fluctuations, almost daily, in the market values of gold, with unusual changes in the current values of stocks. %
  The advices as to the war movements are of a most satisfactory kind, leading to a fall in gold from 146 1/2 at the close of June, to 123 1/2, the 20th inst., at which dates the public had learned the  capitulation of Vicksburgh and of Port Hudson, the defeat of Lee in  Maryland and Pennsylvania, and of successful results at other points  for the Union forces. %
\parencite[159]{BankersMagazine1864}
\end{quote}



\subsection{U.S. Government Bond Yields are Primarily a Function of the Expected Cost of the War}
\label{sec:u.s.-governm-inter}

Since the cash flows of U.S. government bonds are not directly related to a war outcome, in order to treat those yields as an expectation of a war outcome, I need to, first, establish which war outcomes can influence future cash flows, and, second, establish that expectations about war outcomes are the primary factor influencing the yield.
The U.S. government bond yields are plausibly related to both expectations about whether the U.S. would win the war, as well as the cost of the war to the U.S. government.
These two expectations have possibly cross-cutting implications on bond yields, with a increased probability of U.S.\ victory positive (lower yields), but an increased cost negative (higher yields).
Unfortunately, it is not possible to separate these effects using only U.S. government bond yield data.
However, it is likely that U.S.\ government bond yields mostly responded to expectations about the cost of the war, and thus, can be considered a proxy for the expected cost of the war to the U.S.

The riskiness of U.S. government bonds depended on expectations about the cost of the war to the U.S. since the war was funded primarily through debt.
Thus, expectations of a longer or more vigorously fought war almost directly imply and increase in government debt.
The U.S. primarily paid for the war using the debt.
The proportions of the U.S. government revenues for each source of funding as follows: taxes 72--89\% from loans, 11--25\% from taxes, and 0--4\% from miscellaneous sources (tariffs, land sales).
In its analysis of the future fiscal situation, the 1863 \textit{Annual Report of the Treasury} notes that taxes and other revenue sources were covering the non-war expenses and debt payments in its budget, but new debt was being issued to pay for the military expenditures needed to wage the war \parencite[10-13]{Treasury1863}.
\footnote{See also \textcite[][14]{Godfrey1976}.}
The budgetary effect of the war is clear from  comparing forecasts of the U.S. budget for the 1866 fiscal year made before and after the war had ended.
In \textit{The Annual Report of the  Treasury} issued on December 6, 1864, the forecast deficit for FY 1866, assuming the war continued, was \$470 million, with \$1,168 million in expenditures \parencite[13]{Treasury1864}.
In the following year's \textit{Annual Report} issued in December, 1865, the first after the conclusion of the war, the Treasury estimated a surplus of \$112 million, with expenditures of only \$396 million dollars \parencite{Treasury1865}.
This forecast was close to what would be the actual surplus of \$132 million \parencite[2]{Treasury1866}.
Thus, both expectations of the government military expenditures and the length of the war should have had a direct impacts on investors' expectations of the eventual debt level of the U.S. government.
Since increased debt would increase the possibility of default, it would make the bonds riskier, and the yields higher.

The future resources of the U.S. to pay off that debt also depended on the result of the war, namely, whether the seceding states would be reincorporated into the U.S. and potentially help pay for the costs incurred in the war.
The size and wealth of the seceding Southern states was non-trivial, but still much smaller than those of the Union states.
In 1860, the states that would make up the Confederacy accounted for 29 percent of the overall U.S. population,%
\footnote{
  Southern states included VA, TN, GA, NC, AL, MS, LA, SC, TX, AR, FL \textcite[5]{Eicher2001}.
  Wealth is used instead of
}
and 25 percent of the total wealth.%
\footnote{
  \textcite[12]{Elder1865}. Wealth is defined as the value personal property excluding slaves, which was asked on the 1860 Census.
  It may be a better measure of how contemporaries would have viewed the value of the South than GDP because GDP had not yet been invented.
  The closest contemporaries had to GDP was an approximation that the value of yearly production was about 25-27 percent of total wealth \parencites[7]{Elder1865}[24]{Treasury1865}.
}
% TODO: What was the Confederacy GDP?

Some statements by government officials and market participants suggest that they are more concerned with the cost and duration of the war than with the the reincorporation of the seceding states.
William Elder, a leading economist of the era, in an 1863 analysis of the ability of the U.S. to repay its debt stated ``The Rebellion leaves our capital in real and personal property just where it was before the secession. \dots{} So far now this is only a loss of that which we have not had, and at best or worst, a very small one in any time of need'' \parencite[19]{Elder1863}.%
\footnote{
  However, \textcite{Elder1863} does emphasize the importance of the resources of the Western territories, which perhaps strengthens the argument of  \textcite{Weingast1998} that he U.S. was concerned that Southern states seceding could lead to Western states seceding.
}
The \textit{Annual Report of the Treasury Department} in 1863 highlighted this relationship between cost of the war and the interest rate it paid on its bonds:
\begin{quote}
It will not escape observation that the average [interest] rate is now increasing, and it is obvious that it must continue to increase with the increase of the proportion of the interest-bearing to the non-interest-bearing debt.
And as the amount of the latter, consisting of United States notes and fractional currency, cannot be materially augmented without evil consequences of the most serious character, the rate of interest must increase with the debt, and approach continually the highest average.
That must be greater or less in proportion to the duration and cost of the war. \parencite[13]{Treasury1863}
\end{quote}

The striking difference in the trend of the prices of U.S. bonds and that of the Confederate currency (Graybacks) in Figure~\ref{bonds:fig:fives1874_compared} is evidence that U.S. bond yields were not driven by expectations of U.S. victory, and, by elimination, likely represent the cost of war to the U.S.
The price of the Confederate currency decreases almost monotonically over the entire course of the war.
Although the price of Graybacks is partially determined by the money supply \parencite{BurdekinWeidenmier2001}, the price of Confederate bonds issued in Amsterdam and payable in gold followed a similar nearly monotonically decreasing trend mid-1863 onward \parencite{HaberMitchenerOosterlinckEtAl2015}.
The prices of the Confederate assets were likely primarily determined by whether the U.S. would win the war and repudiate their debt \parencite{HaberMitchenerOosterlinckEtAl2015}.
If the U.S. bonds reflected the probability of U.S. victory, then the expected trend of its prices (yields) would increase (decrease) immediately after the start of the war, and almost monotonically increase (decrease) throughout the war, or at least from mid-1863 onward.
However, Figure~\ref{bonds:fig:fives1874_yield} shows that while the yields of U.S. government bonds fluctuates early in the war, they do not decrease over the course of the war.
Instead, they spike in mid-1864 at a point which U.S. victory should have been highly probable if not almost certain.
This suggests that U.S. government interest rates are not primarily reflecting expectations of the war result, and are, instead, reflecting some other factor, which is likely the cost of the war.

Finally, the financial benefit of victory and the costs and duration of war are inversely related.
The longer and more destructive the war, the less the ability of the Southern states to provide for the repayment of the debt incurred by the U.S. in fighting the war.
In fact, immediately after the war, the U.S. government attempted to collect direct taxes from the Southern states, but it soon suspended the collection of the direct taxes after it became clear that the Southern states could not pay them.
Treasury Secretary Hugh McCullough justified this suspension by noting that the inhabitants of the Southern states ``had been subject to heavy taxation by the government which was attempted to be established in opposition to that of the United States, and had been greatly exhausted by the ravages of war.'' \parencite[29]{Treasury1865}.
In analyses of the repayment of debt after the war by the Treasury Department and others \parencites{Elder1865}{Treasury1865}{Walker1865a}, the wealth of the Confederate states was assumed to remain constant between 1860 and 1870 due to the destruction of the war, and that they would not be able to contribute to the repayment of the debt until after 1870.

% todo: longer war reduced the pie. Confederacy in worse shape to repay the debt. E.g. ability to extract a direct tax.

While the precise meaning of the bond yields in terms of a war outcome is unclear, it is clear that the bond yields reflect a war outcome and are being driven almost completely by expectations of the war.
As noted earlier in this section, military expenses accounted for the overwhelming majority of U.S. government expenditures between 1861 and 1865, so there are no other budgetary concerns that could plausibly affecting investor's expectations.
The importance of the war can also be seen by comparing the level and variance of U.S. government bond yields during the war with the preceding period.
During 1856--1860 yields on U.S. government bonds were stable at below 5 percent.
At the height of the the Panic of 1857, the yield of U.S. government bonds, the Sixes of 1868, only reached 5.1 percent.%
\footnote{Source: Data from the \textit{Bankers' Magazine} and the author's calculations. Also see \textcite{HomerSylla2005}.}
Since a major financial crisis had little effect on bond yields prior to the war, there are certainly no non-war omitted factors affecting the yields.
This is not to say that the battles in the war are the only factor affecting the bond yields.
Government policies, elections, and other events could and almost certainly had some affect on bond yields.
But all of these other events were affecting the government bond yields through their effects on expectations of investors' of the eventual outcome of the war.

In summary, the U.S. government bond yields reflected some combination of expectations about the cost, duration and the victor of the war, weighted by how each of these would affect the ability of the U.S. government to repay its debts.
However, it is more likely that these yields are largely reflecting expectations about the cost and duration of the war.

\begin{figure}[!htpb]
  \centering
  \begin{subfigure}[b]{\linewidth}
   \includegraphics{../bonds_and_battles//paper/figure/fives1874_price-1}
  \caption{Price in gold dollars. Par value is \$100.}
  \label{bonds:fig:fives1874_price}
\end{subfigure}
\begin{subfigure}[b]{\linewidth}
   \includegraphics{../bonds_and_battles//paper/figure/fives1874_yield-1}
  \caption{Yield to maturity.}
  \label{bonds:fig:fives1874_yield}
\end{subfigure}
\caption[Price and yield of the Fives of 1874, 1858-1865]{Price and yield of the Fives of 1874 from its issue in 1858 until the end of the Civil War in 1865.
The yield is steady at near 5 percent until the election of Lincoln and the secession of the Southern States in late 1860-early 1861.
During the American Civil War, the yields are variable, spiking in the middle of 1864.
 }
\label{bonds:fig:fives1874_yield_price}
\end{figure}

\begin{figure}[!htpb]
  \begin{subfigure}[b]{0.45\linewidth}
    \includegraphics{../bonds_and_battles//paper/figure/fives1874_greenback_price-1}
  \caption{Price in gold dollars for the Fives of 1874 compared to the price of Greenbacks, 1862-1865.}
  \label{bonds:fig:fives1874_greenbacks}
\end{subfigure}%
\hspace{0.1\linewidth}%
\begin{subfigure}[b]{0.45\linewidth}
    \includegraphics{../bonds_and_battles//paper/figure/fives1874_union-1}
    \caption{Yield to maturity of the Fives of 1874 compared with other U.S. bonds.}
  \label{bonds:fig:fives1874_union}
\end{subfigure}

%% \begin{subfigure}[b]{0.45\linewidth}
%%   \centering
%%   \includegraphics{file.path(fig_path, "fives1874_north-1")}
%% \caption{Yield to maturity of the Fives of 1874 compared with bonds issued by Northern states.}
%% \label{bonds:fig:fives1874_north}
%% \end{subfigure}%
%% \hspace{0.1\linewidth}%
%% \begin{subfigure}[b]{0.45\linewidth}
%%   \includegraphics{file.path(fig_path, "fives1874_south-1")}
%% \caption{Yield to maturity of the Fives of 1874 compared with bonds issued by Southern and border states.}
%% \label{bonds:fig:fives1874_south}
%% \end{subfigure}

\begin{subfigure}[b]{0.45\linewidth}
  \includegraphics{../bonds_and_battles//paper/figure/fives1874_graybacks-1}
\caption{Price in gold dollars of the Fives of 1874 compared with \$100 Graybacks (Confederate currency).}
\label{bonds:fig:fives1874_grayback}
\end{subfigure}
\caption[The Fives of 1874 compared with other government and states issued bonds and currency.]{The Fives of 1874 compared with other government and states issued bonds and currency.
  The Fives of 1874 follow a similar pattern as that of other U.S. bonds and those issued by Northern states.
  They diverge and have a different pattern than the Confederate currency and bonds issued by Southern states.
}
\label{bonds:fig:fives1874_compared}
\end{figure}

% TODO: this needs qualitative overview of the


\section{Battle Data}
\label{sec:battle-data}

A theoretical and practical issue in the empirical study of war is how to operationalize military combat within wars.%
\footnote{
  See the discussions in \textcites{ReiterStam1998}{BiddleLong2004}{Reiter2003}{GrauerHorowitz2012}{Weisiger2015}.
}
There are two major issues in the operationalization of military combat.
The first is ``what the unit of observation?''.
Some work uses events, such as battles or campaigns, as the unit of observation \parencites{ReiterStam1998}{ReiterStam2002}{Ramsay2008}[59-60]{Reiter2009}{PilsterBoehmelt2011a}{GrauerHorowitz2012}, while other work uses time-period aggregated values, \eg{} monthly casualties \textcite{Weisiger2015}.
The second is ``how to measure military success?''.
Some work uses the categorical measures of victory or defeat in battle \parencites{ReiterStam1998}{ReiterStam2002}{GrauerHorowitz2012}, while other work uses casualty based measures, such as the loss-exchange ratio  \parencites{Biddle2004}{BiddleLong2004}{PilsterBoehmelt2011a}.
In this work, quantify military combat in the American Civil War by focusing my analysis on a set of ``militarily significant'' battles, and define success in combat by a categorical coding of their result: Union or Confederate victory.
An advantage of this method is that it allows me to effects individual battles without fewer assumptions of how features of those battles relate to war outcomes, and as well as the average effect of Union and Confederate battle victories.

The battles used in the analysis are listed in Table~\ref{bonds:tab:battles}.
This is the set of the battles which the National Park Service's Civil War Sites Advisory Commission (CWSAC) defined as the most militarily significant in the war, which they defined as ``having a decisive influence on a campaign and a direct impact on the course of the war'' \parencite{CWSAC1993}.
These data come from the 1993 CWSAC report, Civil War Sites Advisory Commission Report on the Nation's Civil War Battlefields \parencites{CWSAC1993}{CWSAC1993b}.
The CWSAC was established in accordance with a 1990 law to preserve Civil War battlefields and was ``asked to identify the nation's historically significant Civil War sites; determine their relative importance; determine their condition; assess threats to their integrity; and recommend alternatives for preserving and interpreting them.'' \parencite{CWSAC1993b}
As part of that process they identified 383 battles, of the over 10,500 military engagements listed in the \textit{Official Records of the American Civil War}, as the ``principal battles'' of the war based on a ``military significance'' criteria.
As to the comprehensiveness of their list, \parencite{CWSAC1993} states,
\begin{quote}
  These [battle] sites encompass virtually all of the principal land battles that were of special strategic, tactical, or thematic importance to local operations, campaigns, theaters, or to the war as a whole \dots{}
  The more than 10,500 conflict sites excluded from our inventory were relatively unimportant as individual military actions.
  These conflicts were the venues and actions that implemented the war between and beyond the dramatic major engagements.
\end{quote}
These principal battles were further classified into four categories of military significance: A (``having a decisive influence on a campaign and a direct impact on the course of the war'') to D (``having a limited influence on the outcome of their campaign or operation but achieving or affecting important local objectives'').
This work will use  43 battles with the ``A'' military significance classification.%
\footnote{
  Three CWSAC class ``A'' battles are omitted from the analysis.
  The Battle of Fort Sumter (SC, April 12--14, 1861) is omitted because it marks the start of the war and the analysis starts after this battle.
  Given the small size of Fort Sumter, no casualties, the effects of the battle can be attributed to marking the start of conflict.
  The Battle of Fort Blakely (AL, April 2--9, 1865) is omitted because news of battle does not reach New York City until after news of the surrender of Gen. Robert E. Lee's forces at Appomattox Court House on Apr 10, 1865.
  The Battle of Five Forks (VA, April 1, 1865) is omitted because I could not find reports of it in the \textit{New York Times}; news of it was overshadowed by Third Battle of Petersburg on April 2nd, which ended the Siege of Petersburg.
}
While this is a subset of the battles, it is still a relatively large number of battles by most war standards and similar in number and composition as the list of battles in other sources: \textcite{Livermore1900} includes 63 battles, \textcite{Bodart1908} includes 50, \textcite{cdb90} includes 49.
Although this is the may be considered a subset of battle of the American Civil War, that is largely due to the completeness of the records and richness of the data on the American Civil War.

\begin{figure}[!htpb]
  \centering
  \includegraphics{../bonds_and_battles//paper/figure/fig_major_battles-1}
  \caption{Major battles and their outcomes}
  \label{bonds:fig:major_battles}
\end{figure}

The CWSAC report \parencite{CWSAC1993b} gives each battle a result, which is one of ``Union victory'', ``Confederate victory'', and ``Inconclusive''.
While \textcite{CWSAC1993} does not explicitly describe the methodology used to classify the battle outcomes, the classifications largely agree with other sources \parencites{fox1898regimental}{Livermore1900}{Bodart1908}{cdb90}, several of which are themselves cited as sources in the report.
Two of those sources, \textcite{fox1898regimental} and \textcite{Livermore1900}, provide more description of how they assigning victory.
Generally, the victorious side of a battle is considered to be the side that controls the battlefield at the end of battle.
The data used in this work includes 30 Union victories, 13 Confederate victories, and two ``Inconclusive'' battles.
For the analysis, the two ``Inconclusive'' battles,  The Wilderness, and Spotsylvania Court House, will be recoded since with only two battles, I cannot estimate an average effect of ``Inconclusive'' battles.
The Wilderness was reclassified as a Union victory, and Spotsylvania Court House as a Confederate victory based on how the results were reported in the news.
Figure~\ref{bonds:fig:major_battles} plots the date and outcomes of these battles over time.%
Since these by selecting on the most militarily significant battles, I have already effectively selected those battles with the highest casualties and force sizes, I will not explicitly include casualty or force size related variables in the analysis.

\begin{table}
  \centering
  \input{../bonds_and_battles//paper/tex/tab_major_battles.tex}
  \caption{List of the 43 major battles of the American Civil War included in this analysis.}
  \label{bonds:tab:battles}
\end{table}

The final data that I need in order to estimate the effects of these battles on bond yields are the times at which the New York markets received news about these battles.
This is not the same as the date of the battle, but occurs with some lag.
But in general, New York markets received news about battles surprisingly quickly:
\begin{quote}
Members of both Houses [of Congress], and of all political creeds, resident bankers, the lobby agents, clerks, and secretaries, haunted the War Department for the latest news from the seat of war.
The daily registry of the Gold Room was a quicker messenger of successes or defeats than the tardier telegrams of the Associated Press. A private secretary of a high official, with no capital at all save his position, which gave him authentic information of every shaping of the chess game of war a full twenty hours in advance of the public, simply flashed the words ``sell, buy'' across the wires, and trusted to the honor of his broker for the rest. \parencite[245]{Medbery1870a}\footnote{Originally quoted in \textcite{WillardGuinnaneEtAl1996}.}
\end{quote}
However, there was still a lag between the occurrence of the battle, and when news of the outcome.
More importantly for the analysis, this lag varied with the distance of the battle from New York and Washington, as well as some idiosyncratic circumstances.
For example, the market knew nothing of Sherman's army after he moved north from Savannah until he arrive d  to the sea until he arrived  \textcite[204]{Mitchell1903}.
As a measure of when information of the battle reaches the New York market, I coded when news about a battle was first reported in the \textit{New York Times}.
This was not necessarily one day, for some battles (Chattanooga) which lasted several days of approximately equal intensity, I coded news reaching the market on several days.
However, for sieges which could last many days, \eg{}Vicksburg, Port Hudson, I only coded when news of the end of the siege was reported.
While battles the Eastern theater were reported on the same or next day (Antietam, Gettysburg), news on battles in other theaters may not reach the market for days.
The battle with the longest lag in news was the Battle of Glorietta Pass in New Mexico; it ended on March 28, 1862, but news of the battle did not reach New York city until April 15.
A data driven approach to identifying the timing of this news is important since in a time series analysis, the timing of the events is the primary factor in identifying their effects.
% Illustrative anecdotes: McClellan in Peninsula campaign, Sherman in march to the sea.

In using a small set of battles based on a \textit{ex post} expert assessment of their military importance, it is important to note that this is not selecting on the dependent variable.
In this case, that would be including explanatory variables by selecting periods of times that had large changes in yields.
Instead, i am selecting a set of covariates, \ie{} battles, which are expected to have large effects on the response variable based on prior information.
This is no different than any other regression model, in which only a small set of covariates are selected of all possible variables in the universe, because those covariates are \textit{a priori} expected to plausibly have effect on the response variable.

For this project, I created a public data collection of American Civil War battles, the American Civil War Battle Database (ACWBD) \parencite{Arnold2015b}.%
\footnote{The data and source code of the data collection are available from \url{https://github.com/jrnold/acw_battle_data}.}
The ACWBD combines, cleans, cross-references battle data from multiple primary and secondary sources, including \textcites{Phisterer1883}{Livermore1900}{Bodart1908}{dyer1908_war_rebel}{KennedyConservation1998}{CWSAC1993}{cwsac2012}.
It is the most complete, machine-readable data on American Civil War battles of which I am aware.



\section{Statistical Models of the Effects of Battles on Bond Yields}
\label{sec:model-war-events}

In this section, I present several models of the effects of battles on U.S. government bond yields.
As noted in Section~\ref{sec:5s-1874}, the yields of Fives of 1874 will be used as the government bond.
In this section I estimate four models of bond yields:
\begin{itemize}
\item Model 1 ($\Model{1}$): A benchmark model of the yields not including any battles.
\item Model 2 ($\Model{2}$: Includes variables for Confederate and Union victories, but no individual battle effects.
\item Model 3 ($\Model{3}$): In addition to variables for Confederate and Union victories, it includes random effects for individual battles.
\item Model 4 ($\Model{4}$): Similar to $\Model{2}$, but the effect of Confederate and Union victories is smoothly varying over time.
\end{itemize}
For these models, I compare the estimated average effects of Confederate and Union victories, the effects of individual battles, and the effect of battles on yields.

The response variable is the yields of the Fives of 1875, as discussed in Section \ref{sec:5s-1874}.
The are  observations between 1861-04-27, the first observation after the Battle of Fort Sumter and the official start of the war, and 1865-05-04, the end of the month after Robert E. Lee surrendered after the Battle of Appomattox Court House.
These are observed at irregular frequencies, 1 to 1 days, but the estimation procedure handles missing values.
In the models, I use the logarithm of the yield since Figure \ref{bonds:fig:fives1874_yield_price} suggests that the variance of yields is increasing with price.

All the models will estimate the effect of battles on the difference in log-yields allowing for measurement error in the observation of those yields, for days $t = 1, \dots, T$,
\begin{align}
  \label{bonds:eq:6}
  \log y_{t} &= y^{*}_{t} + \epsilon_{t} & \epsilon_{t} &\sim \dnorm{0, \tau^{2}} & \text{if $y_t$ not missing} \\
  \label{bonds:eq:1}
  y^{*}_{t} &= y^{*}_{t - 1} + \delta_{t} + \eta_{t} & \eta_{t} &\sim \dt{5}{0, \sigma}
\end{align}
Equation~\eqref{bonds:eq:6} states that observed log-yields are observed with error, and $y^{*}$ is a latent variable representing the ``true'' log-yields.
Equation~\eqref{bonds:eq:1} states that the change in the latent log-yields, ($y^{*}_{t} - y^{*}_{t-1}$, is a function of changes in that periods, $\delta_{t}$, and mean-zero innovation, $\eta_{t}$.
The innovation, $\eta_{t}$, is distributed Student's $t$ to allow for large events and changes that are not modeled in $\delta_{t}$.
If $\tau^{2} = 0$, then Equations~(\ref{bonds:eq:6}) and~(\ref{bonds:eq:1}) would be equivalent to simply estimating the a model with the first difference of $y$ as the dependent variable.
With a non-zero $\tau$, and supposing $\omega$ was distributed normal, this would be equivalent to a constrained $ARIMA(0, 1, 1)$ model \parencite[91]{PetrisPetroneEtAl2009}.


The models differ in how $\delta_{t}$ is defined.
Model $\Model{1}$ includes no battle effects, so $\delta_{t} = 0$ for all $t$.
In the other models, $\delta_{t}$ is non-zero and time-varying, and is function of battle outcomes.
The other model differ in how those battle outcomes are modeled.

In model $\Model{2}$, $\delta_{t}$  includes the effects of Confederate and Union victories,
\begin{align}
  \label{bonds:eq:2}
  \delta_{t} &= \alpha + X_{t} \begin{bmatrix}\beta_{C} & \beta_{U}\end{bmatrix}'  \\
  X_{t, 1} &= \sum_{b \in B} W_{b,t} && \text{if $b$ is a Confederate victory} \\
  X_{t, 2} &= \sum_{b \in B} W_{b,t} && \text{if $b$ is a Union victory}
\end{align}
where $\alpha$ is a constant intercept, $X_{t}$ is an $n \times 2$ matrix containing the sum of the weighted indicators for all Confederate and Union victories occurring on day $t$, $B$ is the set of battles, and $\beta$ is a length 2 row vector parameter for the average effects of Confederate and Union victories.
The matrix $W$ contains the weighted indicators of when news about a battle reaches the market.
The elements are non-zero if the news of the battle is coded as reached the market on that day, and zero otherwise, as discussed in Section~\ref{sec:battle-data}.
News of battles is assumed to be present on the days the first news of the battle is reported in the \textit{New York Times} and the next 5 days.
For a given battle, each day in which news is received is given an equal weight, and the weights for each battle sum to one.
For example, the end of Gettysburg was reported in the news on July 2nd--4th, 1862, so July 2, 1862 through July 9, 1862 have a weight of $\frac{1}{8}$, and all other days have a weight of 0.
The matrix $X$ sums all the weights for all battles of a given outcome.
Since the weights of each battle sum to one, $\beta_{C}$ is interpreted as the average effect of a single Confederate victory, and $\beta_{U}$ as the average effect of a single Union victory.

In model $\Model{3}$, $\delta$ also includes random effects for each battle, in addition to the average effects for Union and Confederate battles,
\begin{align}
  \label{bonds:eq:10}
  \delta_{t} &= \alpha + X_{t} \begin{bmatrix}\beta_{C} & \beta_{U}\end{bmatrix}' + W_{t} \gamma' \\
  \label{bonds:eq:12}
  \gamma_{b} &\sim \dt{7}{0, \zeta^{2}}
\end{align}
where $\alpha$, $\gamma$ and $\beta$ have the same meaning as in $\Model{2}$, and $\gamma$ is a length $B$ vector with a random effect for each battle.
The random effects, $\gamma$, are given a Student's $t$ distribution since it is expected that a few battles may have large effects that are much different than the mean.
The total effect of an individual battle is the average effect for its outcome plus its random effect,
\begin{align}
  \omega_{b} =
  \begin{cases}
    \beta_{C} + \gamma_{b} & \text{if $b$ is a Confederate victory} \\
    \beta_{U} + \gamma_{b} & \text{if $b$ is a Union victory}
  \end{cases}
\end{align}

In model $\Model{4}$, the effects of each battle, $\omega_{b}$ are both shrunk towards the average effect of its outcome and of the effect of previous previous battle of the same result.
This is equivalent to a model in which the effects of Confederate and Union victories are time-varying with an $AR(1)$ process.%
\footnote{
   $\omega_{C,t} \sim N(\beta_{t,C}, \zeta^{2}) N(\omega_{C,t-1}, \xi^{2}) = N((1 - \rho) \beta_{C} + \rho \omega_{C,t-1}, \rho \zeta^{2})$, where $\rho = \frac{\xi^{2}}{\xi^{2} + \zeta^{2}}$.
}
\begin{align}
  \label{bonds:eq:11}
  \delta_{t} &= \alpha + W_{C,t} \omega_{C} + W_{U,t} \omega_{U} \\
  \omega_{C,b} & \sim \dnorm{(1 - \rho) \beta_{C} + \rho \omega_{C,b-1}, \rho \zeta^{2}} & \text{for $b \in \{2, \dots, C\}$} \\
  \omega_{C,1} & \sim \dnorm{ \beta_{C}, \zeta^{2}} \\
  \omega_{U,b} & \sim \dnorm{(1 - \rho) \beta_{U} + \rho \omega_{U,b-1}, \rho \zeta^{2}} & \text{for $b \in \{2, \dots, U \}$} \\
  \omega_{U,1} & \sim \dnorm{ \beta_{U}, \zeta^{2}}
\end{align}
Where $C$ is the number of Confederate victories, $U$ is the number of Union victories.
The Confederate and Union victories are ordered by their dates.
For Confederate victories, the first battle was First Bull Run (July 21, 1861), and the last was the Crater (July 30, 1864).
For Union victories, the first battle was Fort Donelson (February 11--16, 1862), and the last was Appomattox Court House (April 9, 1865).
This means that the effect of Gettysburg, $\omega_{13}$ would be shrunk towards the average effect of all Union victories, $(\beta_{U}$, and the effect of the previous battle, $\omega_{12}$ (Champion Hill).%

In all these models, since $y^{*}$ represents log-yields, the parameters $\beta$, $\gamma$ and $\omega$ can be interpreted as approximately the percent change in the yield due to the outcome of a battle.

There are two things to note about the models that I have presented in this section.
I do not include variables for other events in the war.
While changes in policy, leadership, foreign policy events, negotiations and other events certainly influence expectations about the future of the war, I do not have a coherent, non \textit{ad hoc} method for choosing a set of events.
\textcite{McCandless1996} and \textcite{SmithSmith1997} both include lists of indicator variables for \textit{ad hoc} lists of military, fiscal, and political events in the war.
However, there is no clear rule as to what constitutes major non-battle events in the war, and what does not.
In fact, \textcite{SmithSmith1997} includes indicator variables for a list of war, and political and financial news, but these are generated by selecting on the dependent variable, by using events mentioned in \textcite{Mitchell1903} as corresponding to large daily movements in the gold market.%
\footnote{
  An exception to this is funding bills, where their effect can be modeled as a function of how much each bill increases the debt.
  The difficulty of including funding bills is the effect on the market should not occur when the bill is signed, which is easy, but during the negotiations of the bill when information about the possible outcome of the vote is known.
}
Excluding these non-battlefield events may not be an issue because they are more likely influenced by battlefield events than vice versa.
For example, the capture of Atlanta is given as a reason why Lincoln won the election of 1864, and the victory at Antietam encouraged Lincoln to sign the Emancipation Proclamation.
The outcomes of battles an influence expectations about the entire future course of the war, including non-battle events.
Thus, the effects of the non-battle events on the prices and yields of financial assets will only incorporate the surprising part, are only the events that are not anticipated given the previous events on the battlefield.
My claim is that battle outcomes are more random, and thus more surprising than policy events.
In Section~\ref{sec:conclusion}, I propose a more coherent method for identifying war and non-war events.

Unlike many analyses of bond yields, the models presented here do not include either inflation or a risk-free interest rate.
Inflation is not included for several reasons.
First, while inflation measured in U.S. currency (Greenbacks) rose during this period, since the price of gold closely tracked the price of commodities during this period, inflation in terms of gold dollars was low\parencites{Mitchell1903}{Mitchell1908}.
Second, even if that were not the case, the influence of the war was such that inflation would likely be considered a post-treatment variable that, like bond yields, was responding to war events.
Third, the best measure of inflation, the Warren-Pearson Wholesale Price Index, is only available at the monthly level \parencite{WarrenPearson1933a}.
The risk free interest rate during this period is not used because there was an asset that was clearly risk free during this time \parencites{HomerSylla2005}.
In later periods, U.S. government bonds would be considered a risk free interest rate.
But this work uses U.S. government bonds precisely because they are risky.
\textcites{Macaulay1938}{HomerSylla2005} consider indexes of railroad bonds and New England municipal and state bonds as risk free assets during this period.
During this period, railroad bonds were still relatively risky during this period, and had higher yields than U.S. government bonds.
Municipal and New England state bonds, when their yields are calculated in gold dollars, had yields similar to those of U.S. government bonds.
Also, since the New England states were issuing bonds to pay for their war expenditures, the yields of these were also linked to war outcomes.
Additionally, the only currently available comprehensive source of these interest rates, \textcite{Macaulay1938}, provides these interest rates at a monthly (railroad) and quarterly (New England state and municipal bonds) frequency.
Other work in 19th century economic history uses the interest rate of British consols as a risk free rate \parencite{BordoRockoff1996a}.
This is inappropriate for this analysis because that risk free interest rate was for the London market, and not the New York market.
Additionally, during the war there was the possibility that the United Kingdom would enter the war, so consols may not have also incorporated war-risk.


All models are estimated using MCMC methods to sample from the posterior distribution.
Samples are drawn using the NUTS-HMC algorithm \parencite{HoffmanGelman2013} as implemented in the probabilistic programming language, \Stan{} \parencites{Stan2015a}.
I found that the sampling efficiency could be improved in estimating these, and other space models, in \Stan{} by using a two step method.
In the first step, I marginalize over the latent states of a state space model,  $y^{*}$ in this case, and use HMC-NUTS to sample the other parameters.
In the second step, given the values of the parameters, I use the forward-filter backwards-sample algorithm to sample the latent states.
This is an example of a partially collapsed Gibbs sampler \parencite{VanDykPark2008a}.
I describe this method and its implementation in \Stan{} in more detail in Chapter~\ref{cha:structural-breaks} of this dissertation.
For all models, four chains were run, and convergence assessed using multiple-chain R hat statistics and the number of effective samples, as recommended in \parencites{Stan2015a}.




\subsection{Average Effects of Union and Confederate Victories}
\label{sec:future-battles}

\begin{table}
  \centering
  \input{../bonds_and_battles//paper/tex/avg_battle_result.tex}
  \caption{Average Effects of Confederate and Union Victories on Log-Yields.}
  \label{bonds:tab:avg_battle_effects}
\end{table}

For model $\Model{2}$, the average effect of a Union victory is the parameter $\beta_{U}$, and the average effect of a Confederate victory is the parameter $\beta_{C}$. 
For models $\Model{3}$ and $\Model{4}$, I use the average of the total battle effects as the average effect.
For these models, the average effect of a Union victory is $\tilde{\beta} = \frac{1}{U}\sum_{b \in U} \omega_{b}$ and the average effect of a Confederate victory is $\frac{1}{C}\sum_{b \in U} \omega_{b}$. 

Table \ref{bonds:tab:avg_battle_effects} shows summary statistics for the average effects for the models including battle-effects.
The estimated avarege effect Confederate victory on yields was positive, and the estimated average effect of Union victories was at best small.
A Confederate victory on average is estimated to increase the yield approximately 3.6--5.3 percent, with the 95\% central credible intervals ranging between 0.036 and 0.086.
A Union victory on average is estimated on average no or little effect on yields, decreasing the yield 0.7--1.5\%, although the 95\% central credible intervals all contain 0.
There are two explanations for this asymmetry in effects of victories on the expected cost of the war.
First, individual Union victories had a smaller effect on the overall cost and duration of the war than Confederate victories.
Second, the market was already expecting Union victories, so they were less surprising than Confederate victories.
That Confederate victories increased the expected cost of the war is consistent with the market expecting a Union victory in the war and discounting the possibility of Confederate victories leading to a quick peaceful resolution.



\subsection{Effects of Individual Battles}
\label{sec:indiv-effects-battl}


Models $\Model{3}$ and $\Model{4}$ estimate effects for individual battles in addition to the average effects of Confederate victories, although they use different prior distributions to regularize these estimates.
Model $\Model{2}$ assumes that all Confederate and Union victories have the same effect.

Figures \ref{bonds:fig:btl_effects_ordered} plots the battle effects, $\omega_{b}$, for only 

\begin{figure}[!htpb]
  \centering
  \includegraphics{../bonds_and_battles//paper/figure/fig_btl_effects-1}
  \caption[Battle Effects on log-yields for all models]{Battle effects, $\omega_{b}$ for all battles for all models, ordered by ending date of the battle.
    Union victories are in dark blue. Confederate victories are in gray.
  }
  \label{bonds:fig:btl_effects}
\end{figure}


\begin{figure}[!htpb]
  \centering
  \includegraphics{../bonds_and_battles//paper/figure/fig_btl_effects_ordered-1}  
  \caption[Battle Effects on log-yields for model $\Model{3}$]{
    Battle effects, $\omega_{b}$ for all battles for model $\Model{3}$, ordered from largest to smallest.
    Union victories are in dark blue. Confederate victories are in gray.
  }
  \label{bonds:fig:btl_effects_ordered}
\end{figure}



\subsection{Overall Effects of Battles on Yields}
\label{sec:over-effects-battl}


\begin{table}[!htpb]
  \centering
  \input{../bonds_and_battles//paper/tex/tab_waic.tex}
  \caption{
    Predictive criteria for each Widely Applicable Inforation Criterion (WAIC).
    RMSE is the root mean squared error of each model.
    $\text{elpd}_{\text{WAIC}}$ is the expected log posterior predictive density implied by the Widely Applicable Inforation Criterion (WAIC).
  }
  \label{bonds:tab:waic}
\end{table}

\begin{table}[!htpb]
  \centering
  \input{../bonds_and_battles//paper/tex/tab_btl_effects_size.tex}
  \caption{Size of average battle effect compared to the average innovation}
\end{table}


% \begin{figure}
%   \centering
%   \includegraphics{file.path(fig_path, "cum_btl_effects-1")}
%   \caption{Cumulative effects of battles on the yield.}
% \end{figure}

% \begin{figure}
%   \centering
%   \includegraphics{file.path(fig_path, "theta_star-1")}
%   \caption{Estimated yields after removing the effects of battles.}
% \end{figure}



\section{Conclusion and Future Work}
\label{sec:conclusion}


I conclude with some thoughts on future directions of this research.
There are several methodological issues that this analysis had to confront: (1) ``When did the market receive information about battles?'', (2) ``What information about battles did it receive?'', and (3) ``What other non-battle information did it receive?''.
In this analysis, I handled the information about battles by choosing a set of important battles and their results, both of which were coded \textit{ex post} by historians,  with the assumption that these would be a noisy measure of the the market had observe.
I handled when the market received information about battles with data, the day on which the battle was first reported in newspapers, and modeling choices, by including lagged terms.
I handled other non-battle information only by using robust fat-tailed distributions in the analysis, and not through gathering other data.
The reason for this was that coding indicators for alternative events is difficult to do in a non-arbitrary manner.
The models presented in this work could all be improved to better account for these, but that would increase model complexity.
Rather than complicate the model, it is better to get better data.

Directly using the text of contemporary newspapers as data would simultaneously address all of the methodological issues raised in the previous paragraph.
A rough sketch of the analysis would be to estimate topics from the text and regress those topics on market movements.
What information the market received about battles would correspond to battle-related topics inferred from the articles themselves.
When the market received information is simply the days on which those topics appeared in newspapers.
Finally, the non-war related topics provide control variables without having to explicitly specify a set of variables.
The importance of events relative to other events is identified through their prominence in the newspapers, \eg{}if the front page is about a battle, then it is probably the news that is moving the market, whereas if the frontage is about the passage of a law or the economy, then that is probably the more important news.
Nor is this approach completely at odds with the type of analysis conducted in this work which uses expert-coded events; that additional information could be included in semi-supervised methods, \eg{}structured topic modeling CITE.
One important caveat, is that this makes the assumption that the newspaper articles sufficiently reflects the information available to the market.
However, it seems less less problematic than the issues discussed earlier.


%%% Local Variables:
%%% TeX-master: dissertation.tex
%%% End:

%  LocalWords:  Gartzke2007 DafoeKelsey2014a Slantchev2012a 'hide' se
%  LocalWords:  tex RData barg parencites Peace'' textcite cha financ
%  LocalWords:  outcome'' result'' GuidolinLaFerrara2010 Lau2006 premia
%  LocalWords:  HaberMitchenerOosterlinckEtAl2015 RhodeStrumpf2004a '
%  LocalWords:  WolfersZitzewitz2004 Intrade Tradesports Osama CDB90
%  LocalWords:  LeighWolfersEtAl2003 Slantcheve2012a Hall2004 Uppsala
%  LocalWords:  McCandless1996 ChenSiems2004 SchneiderTroeger2006 pre
%  LocalWords:  WolfersZitzewitz2009 Fearon1995 Reiter2003 Powell2004
%  LocalWords:  FilsonWerner2002 Slantchev2003 SmithStam2004 cdb90 eq
%  LocalWords:  LeventogluSlantchev2007 LangloisLanglois2009 ACLED m1
%  LocalWords:  WolfordReiterCarrubba2011 belligerents' ESOC US1901
%  LocalWords:  SarkeesWayman2010 BiddleLong2004 Clodfelter2008 Fives
%  LocalWords:  CochranLong2014 RaleighLinkeHegreEtAl2010 Reiter2009
%  LocalWords:  Goemans2000 WillardGuinnaneEtAl1996 north2000introd
%  LocalWords:  FreyKucher2000 FreyKucher2000a Cornwallis1879 consols
%  LocalWords:  Mitchell1903 investors' Ramsay2008 Weisiger2015 ACWFD
%  LocalWords:  Keele2015a Wagner2000 Biddle2004 misperceptions OKFN
%  LocalWords:  Walter2009 Mitchell1908 Calomiris1988 SmithSmith1997
%  LocalWords:  Schwab1901 Weidenmier2002 BurdekinLangdana1993 sumter
%  LocalWords:  DavisPecquet1990 BrownBurdekin2000 Roll1972 Weiss1966
%  LocalWords:  OosterlinckWeidenmier2007 generalizability Poast2012
%  LocalWords:  Fuller1956a multimanned kalyvas2010inter Bayley1882
%  LocalWords:  BoltZanden2013 GoldinLewis1975 McPherson2003 maxdate
%  LocalWords:  hacker2011census Livermore1900 DeKnight1900 Dewey1918
%  LocalWords:  Treasury1863 HomerSylla2005 fives1874 Elder1863 Sixes
%  LocalWords:  Bankers' Merchants' Macaulay1938 Arnold2015a 20's HMC
%  LocalWords:  40's 30's Ball1991 Treasury1861a Treasury1861b uring
%  LocalWords:  Treasury1862 Treasury1864 Treasury1865 graybacks nrow
%  LocalWords:  BurdekinWeidenmier2001 Vicksburgh BankersMagazine1864
%  LocalWords:  COFEDERACY Godfrey1976 Treasury1866 ReiterStam1998 m2
%  LocalWords:  ReiterStam2002 PilsterBoehmelt2011a CWSAC CWSAC1993
%  LocalWords:  GrauerHorowitz2012 CWSAC1993b Blakely Bodart1908 CWSA
%  LocalWords:  fox1898regimental Spotsylvania Medbery1870a Glorietta
%  LocalWords:  Arnold2015b Phisterer1883 dyer1908 cwsac2012 XXXX sui
%  LocalWords:  KennedyConservation1998 XXXXX MCMC Stan2015a generis
%  LocalWords:  bellum CBO2012 CBO2012a Fives Sixes 20's 40's 30's bodart
%%  LocalWords:  Eicher2001 Elder1865 Weingast1998 Walker1865a sides'
%%  LocalWords:  livermore Powell2006 HistoricalResearchEtAl1984 Fives
%  LocalWords:  Helmbold1995 Shirkey2009a  20's 40's 30's ACWBD won''
%  LocalWords:  ARIMAX Gelman Donelson 'hide' lost'' frac mathcal url
%  LocalWords:  parencite 20's 40's 30's ARIMA PetrisPetroneEtAl2009
%  LocalWords:  diff HoffmanGelman2013 VanDykPark2008a
